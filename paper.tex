\documentclass{article}
\usepackage{url}
\usepackage{mathtools}
\begin{document}
\title{A survey of digital asset representation systems}
\author{Peter Todd}
\date{FIXME}
\maketitle

\section{Attributions}

The reader is cautioned that unfortunately this paper currently lacks proper
citations and attributions due to lack of time, the complexity of determining
the originator for many of these concepts, and because the popular
bitcointalk.org forum where many of these ideas have been discussed was down
due to a hacking incident when much of this paper was written. Pull-requests to
fix this deficiency are welcome.


\section{Introduction}

\section{Design considerations}

\subsection{Issued assets vs. consensus assets}

A distinction must be made between assets issued by a trusted party and assets
whose existance and value is a matter of consensus. The most obvious example of
the former category is directly issued stocks: a company issues the stock and
without that company the asset is meaningless. Assets derived from physical
commodities, such as gold ETFs, are another example: a trusted party must act
as a issuer and guarantor of the asset to provide the mechanism linking the
representation to the underlying asset.  Again, without the trusted third party
the asset is meaningless.

Consensus assets on the other hand exist by virtue of consensus among some
community. Bitcoin itself is an example, as are the domain names registered
with the Namecoin blockchain. A more unusual example is the proposed fidelity
bond mechanism, creating an asset from the act of sacrificing Bitcoins, in
conjuction with the ability to compactly and in a machine-readable way prove
fraud by the owner of the bond. In all these cases the asset only exists in the
minds' of a community who come to consensus that the asset exists and who owns
it through some shared set of mathematical rules.

The options available for representing the asset are more broad when a trusted
issuer is already required: the third-party themselves can maintain and
guarantee the records of who owns what unit of the asset. This is not true with
consensus assets, for which the only known way to record the ledger of assets
in a decentralized fashion is to use a Satoshi-style block chain.


\subsection{Atomic zero-trust transferability}

It should be possible to transfer an asset between owners atomically, without
trusting the other party to fully complete the transfer. However, for
third-party guaranteed systems, it is acceptable to trust them to record the
transfer accurately so long as fraud on their part can be proven.


\subsection{Transparency}

We call the representation of an asset transparent if it is possible for a
third-party to determine who the owner(s) of it are regardless of interference
from the issuer or owner(s). Of course in most cases ownership will be defined
in terms of cryptographic keypairs rather than "meatspace" names, but we talk
about transparency for the purposes of detecting and proving fraud, as well as
more mundane activities like payment of dividends, rather than considerations
of real-world ownership.

A opaque representation does not have this property, however it should have the
property of being provable, allowing the owner of an asset to prove they are
the true owner to a third-party without the assistance of the
issuer.\footnote{If the issuer's assistance is required then why not simply
trust their records, given that the issuer can simply refuse to assist the
owner?}


\subsection{Divisibility}

Many financial assets are divisible, and units of them can be split into parts,
or parts can be merged again. Representing this poses issues, for example if a
fixed ratio of bitcoins to shares is used, a change in the value of either the
shares or the bitcoins representing the shares can cause economic distortions
if the shares are less valuable than the bitcoins. In addition the Bitcoin
system has limits on the minimum transaction output size, known as the "dust
limit". The limit is (currently) based on an assumption that there is no
rational reason to create an output that would cost more to spend in fees that
it's worth, unless you are trying to "abuse" the blockchain by inserting data
into it.\footnote{The addition of the dust limit was triggered in part due to
someone encoding 2.5MiB of Wikileaks data, and materials related to child
abuse, in the UTXO set.}


\subsection{Proof size}

Representations differ in the size of the proofs required to prove their
existence; this issue is present regardless of whether the representation is
transparent or opaque. In most cases each transfer of an asset increases the
proof size by $k$, and merging $j$ units of the asset together results in an a
proof of size $j*k*n$ - like Bitcoin itself this can pose scalability problems.
On the other hand, unlike in the case of Bitcoin, for an asset with an issuer
the issuer can always destroy and re-issue units of the asset to reset the
proof size back to the minimum.

A related issue is whether or not the representation can be verified by
SPV\footnote{Simplified payment verification} clients, and in a transparent
system, whether or not SPV clients can examine all outstanding units of the
asset.


\subsection{Censorship resistance}

A representation is censorship resistant if it is difficult for Bitcoin miners
to determine if a transaction they are mining represents an asset. Of course a
completely censorship proof representation is usually impossible - given the
knowledge that $TXOut_i$ represents an asset a majority of hashing power can
always refuse to mine transactions spending that output - but censorship can be
made to be fairly difficult. In particular censorship of the representation
itself ideally should not be possible.


\subsection{Multi-author/multi-use transactions (e.g. CoinJoin)}

While often the case transactions do not have to have a single author, nor do
they have to have a single purpose. In particular zero-trust asset purchases
for assets represented on the blockchain can be accomplished by having both
parties jointly author a transaction where both the asset, and the funds,
change hands atomicly to their new owners.

In addition representations ideally should be flexible enough that additional
transaction inputs and outputs can be added to the transaction without
interferring with the intended asset transfer. For instance the user may want
to exchange one or more assets for other assets owned by another user
atomically, possibly using different representation systems. Or they may want
to obscure exactly who was paying for what assets by combining the transfer
with the CoinJoin technique of multi-author transactions.

\subsection{Legal}



\section{Asset representation systems}

\subsection{Centrally maintained ledgers}

Here we have the issuer of the asset - or possibly some other third party -
also maintain the ownership ledger. Transfers of the asset can be made atomic
with respect to Bitcoin transactions by having the ledger define a transaction
as valid only if a specific set of transaction outputs exists in the Bitcoin
blockchain. For an asset where the ledger is maintained by the issuer in
principle this is secure: the issuer is already trusted with the asset so
trusting them to maintain the ledger itself adds no additional risks.

A central ledger implementation should incorporate triple-entry-accounting
\cite{triple-entry-accounting} to allow fraud by the issuer in their
maintenance of the ledger to be proven. Consider for example a transaction
where Alice sells Bob $1$ unit of FooCorp stock for $1$ Bitcoin: Alice first
creates a never-before-used Bitcoin address $\mathrm{Addr}_A$ to receive Bob's
funds. Next she creates a transaction stating that if a transaction output
paying $1$ Bitcoin to $\mathrm{Addr}_A$ exists in the Bitcoin blockchain prior
to time $T$ she wishes $1$ unit of FooCorp stock be deducted from her balance
and added to Bob's balance. She signs this transaction and presents it to Ivan,
the maintainer of the FooCorp ledger. Ivan checks Alice's current account
balance, and if she owns the required units of stock Ivan signs the transaction
and gives his signature to Alice. Now she can present it to Bob, and if he is
satisfied that the expiry time\footnote{An expiry time is required in case Bob
fails to complete the transaction; Ivan must deduct all pending transfers from
Alice's balance.} is sufficiently far in the future, he can make the transfer.

Proving fraud varies depending on what exact type of fraud is to be proved, and
mostly follows the same logic as fraud proofs for the Bitcoin
blockchain,\cite{inflation-proofing-via-fraud-notices} as well as off-chain
transaction systems. In particular either all transactions will need to be
public, or the issuer will need to create public commitments to
merkle-sum-trees\cite{merkle-sum-tree} of all balances, so that the total
assets issued can be audited. (much in the same way an off-chain transaction
system would make it possible to audit the funds held in trust by the service)

The chief disadvantage of central ledger systems is that of specialization:
often it is the case that an issuer of some asset might not want to deal with
the overhead, complexity, and high consequences for maintaining the ledger they
are responsible for. In particular the issuer is involved in every buy or sell
of the asset and hence must have a online service to do so; note how in
traditional financial markets it is very common to outsource the task of ledger
maintenance to third-party exchanges, something we wish to avoid in the name of
decentralization.

Central ledger systems do make the most efficient use of blockchain space,
however assets tend to be high-value transactions, so transaction fees are not
likely to be significant for the application.


\subsection{Consensus blockchains}

Like Bitcoin itself we can of course represent assets with a consensus
blockchain, either independently mined, or merge-mined. In addition the
blockchain may take the form of the Bitcoin chain - a linear series of blocks -
or something else entirely like a directed acyclic graph. Regardless of the
exact form the blockchain takes the key issue is how the consensus is achieved;
in the case of a Bitcoin-like proof-of-work chain the state of the chain is
determined by work done. However this leaves the chain vulnerable to attack by
anyone who can do the majority of work, (the 51\% attack) either the sum work
done from the beginning of the system, or simply on an ongoing
basis.\footnote{Note how this means that for the chain to be secure work must
be done continuously.} We now have an incentives problem: why should a given
miner do the work required to mine the blockchain? Namecoin attempted to solve
this problem by conflating the function of Namecoin the distributed consensus
key-value system with Namecoin the currency, resulting in significant
speculation activities. It is not clear what effect this conflation has had on
Namecoin's success; Namecoin has not yet been used in production for its
intended purpose but there are many possible reasons for this failure.

In addition Namecoin allowed merge-mining early in its history. Merge-mining
allows the work done to mine one chain, the master chain, to be re-used for
multiple consensus systems by defining a valid block header as being either a
proof-of-work directly, or a merkle-path leading to a proof-of-work potentially
valid for a different chain. The theory behind allowing merge-mining is that by
making mining essentially cost-free you can encourage those mining the master
chain to merge-mine the secondary chain as well, thus increasing the hashing
power of the secondary chains and thus their resistance to a 51\% attack. If
$P$ is the total hashing power in existance, and $>P/2$ of that hashing power
decides to merge-mine the secondary chain, we can guarantee the security of
that secondary chain the hashing power available to the attacker is
insufficient to launch a 51\% attack.

On the other hand this also perversely means that the \emph{cost} of a 51\%
attack can be negligable. Specifically for any miner already mining the primary
chain, the marginal cost to mine the secondary is nearly zero.  Thus if less
than 50\% of the total hashing power is merge-mining the secondary it may be
possible for an attacker to rent hashing power at low cost. Equally the
attacker may find it profitable to buy their own hashing power, and use it to
both attack the secondary chain, and gain the reward from mining the
primary.\footnote{Note that while merge-mining protocols have all been designed
such that it is always possible to tell \emph{if} hashing power is engaging in
merge-mining they do not force the miner to reveal \emph{which} chains they are
mining, thus allowing a miner to conduct a re-organization attack in secret.}

In the case of Namecoin its hashing power, as a \% of the total Bitcoin hashing
power, has generally remained around FIXME\%, thus allowing large pools to
single-handedly attack Namecoin if they had chosen too. It's easy to imagine a
merge-mined blockchain for the purposes of, say, stock ownership records being
attacked by a government or large corporate entity wishing to destroy the chain
for legal or competitive reasons, and equally it's easy to imagine large
well-known mining pools not wishing to associate themselves with that
merge-mined chain for legal reasons. (other than a government-sponsored attack)


\subsubsection{Atomic transfers in consensus blockchains}

FIXME: \url{https://en.bitcoin.it/wiki/Contracts#Example_5:_Trading_across_chains}


\subsection{Parasitic consensus systems}

A proof-of-work blockchain, such as the Bitcoin blockchain, can be made use of
parasistically for a secondary consensus system. Recall the two fundemental
proofs that a blockchain provides: consensus ordering/timestamping and and
proof-of-publication. A Satoshi-style blockchain can be used as an ordered
message publication service - it is not possible to completely prevent the
publication of data without whitelisting censorship\footnote{By this we mean
    that a majority of miners use whitelists to determine if a transaction is
allowed to be mined.} although publication can be made expensive.\footnote{By
"expensive" we mean the fees required to get the transction mined; see FIXME
for a discussion of data embedding techniques.} Thus for a given block height
$i$ we have a set of blocks $B={b_0 ... b_i}$ containing messages $M={m_0 ...
m_j}$. By applying a fixed set of rules to that set of messages multiple
parties can independently arrive at the same state of the system.

A concrete toy example is the is the "String Bling" system. Here we want to
determine what string $s$ is defined as being the most "blinged", where bling
is gained by provably destroying Bitcoins. In addition we want to make it
possible for a given string $s_1$ to have its bling stolen and transferred to
another string, $s_2$, by a hostile act.

FIXME: rewrite without OP\_RETURN, making bling be abstract

To do this we make use of provably unspendable
outputs\cite{provably_unspendable}: a scriptPubKey whose first operation is
OP\_RETURN can never evaluate true, and thus is trivially proven to be
unspendable. We say that if a transaction with an output containing a
scriptPubKey of the following form exists in the Bitcoin blockchain $v$ BTC
worth of bling have been assigned to string $s$:

OP\_RETURN "BLING" s

We also say that the bling associated with $s_1$ can be transferred to $s_2$ if
a transaction of the following form exists in the Blockchain, provided that the
total bling associated with $s_1$ at the transaction's height is less than $v$:

OP\_RETURN "STEAL" s\_1 s\_2

The algorithm to evaluate the amount of bling associated with all blinged strings is simple:

\begin{verbatim}
S = map()

for block in blocks:
    for tx in block:
        if tx is of form "bling":
            S[s] += tx.value

        else if tx is of form "steal":
            if S[s_1] <= tx.value:
                S[s_2] += S[s_1]
                S[s_1] = 0
\end{verbatim}

Note how there may be transactions in the blockchain that are invalid according
to the rules of the bling system - a transaction ma have fields missing or an
attempt to steal bling may have less value than the bling stolen. The presence
of such transactions is of no concern however as it is the rules, not the
blockchain data to which the rules are applied, that determines of the final
state of the system.

The Mastercoin system uses this principle. While not yet well developed, there
exists an agreed upon set of rules that, from the contents of the Bitcoin
blockchain, can derive a set of "Mastercoin" transactions and a final ledger
state derived from data encoded in the Bitcoin blockchain.

Parasitic consensus systems inherently gain the benefits of the security of the
underlying consensus system. Though the "string bling" system may have only a
handful of users interested in it, an attacker attempting to change the state
of the consensus of what strings have what bling would need to attack the
Bitcoin blockchain directly - a signififantly harder problem. A merge-mined or
independently mined string-bling implementation would probably never be secure
against an attacker with a budget of even just a few thousands dollars; by
parasiticly using the Bitcoin blockchain the attacker\'s required budget swells
to hundreds of millions of dollars.


\subsubsection{Atomic transfers in parasitic consensus systems}

FIXME - basically a non-issue



\subsection{Colored Coins}

For the purposes of asset tracking we can take advantage of the symmetry
between the movement of Bitcoins, and the movement of assets. In both cases we
want to assign assets to owners, and allow those owners to transfer their
assets to new owners.


\subsubsection{Discrete txout scheme}

The most primitive implementation of this idea is to define a specific txout as
representing some particular asset or discrete unit of that asset. For instance
in figure \ref{fig:peter-bond} Peter Todd issued the asset by fiat by signing
a message with his PGP key.\footnote{Peter Todd will honor this agreement.}

\begin{figure}
\label{fig:peter-bond}
{\small
\begin{verbatim}
-----BEGIN PGP SIGNED MESSAGE-----

I, Peter Todd, hereby declare that the following txout and its discrete txout
scheme children is entitled to yearly coupon payments of 0.1BTC for the period
2014 to 2019. Payment shall be made on demand upon the receipt of a message
signed with the scriptPubKey of the txout for the appropriate time period
(pro-rated) stating the address to which the coupon payment shall be sent. Time
shall be defined in terms of Bitcoin block timestamps.

0e3e2357e806b6cdb1f70b54c3a3a17b6714ee1f0e68bebb44a74b1efd512098:0

-----BEGIN PGP SIGNATURE-----

-----END PGP SIGNATURE-----
\end{verbatim}
}
\caption{The PGP-signed statement issuing the asset}
\end{figure}

Figure \ref{fig:peter-bond-transfer} shows the asset being transferred, first
from Satoshi to Alice, in $\mathrm{Tx}_1$, and then from Alice to Bob in
$\mathrm{Tx}_2$. The rule here is very simple: whatever is the first
transaction output is defined as being the asset; the value of the transaction
output is irrelevant, and hence is not shown in the diagram.

Something to keep in mind is that $\mathrm{Tx}_0$'s role is simply to create
the transaction output for the PGP-signed statement to reference;
$\mathrm{Tx}_0$ is \emph{not} what creates the \emph{asset}. Only the
statement, and Peter's ``meatspace'' reputation and honesty, can do that.




Suppose we have a contract $C$ in some agreed upon form. (which may or may not
be machine readable) We can efficiently create a transaction output that in
some way commits to the content of that contract by computing the digest $H(C)$
and embedding that digest into the output's
scriptPubKey:

\begin{verbatim}
    <H(C)> DROP <pubkey> CHECKSIG
\end{verbatim}

Once the txout has been created in a transaction it is not possible to change
the contents of $C$. It is also possible to avoid the
requirement of including the hash in the scriptPubKey separately by using the
homomorphic property of ElGamel-type cryptosystems, thus creating a pubkey that
itself commits to $H(C)$\cite{DBLP:journals/corr/abs-1212-3257}

us proposed\cite{mike-hearn-distributed-markets} by Hearn distributed markets
could be created using external P2P networks to distribute the contents of the
contracts, in conjunction with a special-purpose client. Assets would be
represented by one or more designated transaction outputs, and transactions
would be used to move the asset ownership from one party to another by spending
those transaction outputs.

A small generalization can be made to allow the representation of divisible
assets by using the least significant bits of the txout value fields to match
up transaction inputs previously representing an asset to the outputs that now
represent it.\cite{peter-todd-fidelity-bonds}


\subsection{Colored Coins}

Here the issuer of the asset designates one or more transaction outputs as
representing some amount of the asset. At the most simple, any transaction
output in a transaction that spends those outputs is then defined as also
representing the asset; practical systems need ways to allow some transaction
inputs and outputs in the transaction be left "uncolored" to pay fees and
atomicly conduct sales.




A brief overview: in the Bitcoin system a transaction $T$ consists of an
ordered set of one or more transaction inputs, $I=\{I_0 ... I_n\}$ and a list
of one or more transaction outputs, $O=\{O_0 ... O_n\}$. Transactions are
referred to by their hash, also known as the $\mathrm{txid}$, calculated as
$H(T)$ where $H$ is the SHA256 hash function applied twice. Each input $I_i =
(H(T_j), k, S)$ is a tuple of a previous transaction hash, an index, and a
scriptSig $S$. Each output is a tuple $O_i = (P, v)$ consisting of a
scriptPubKey and value. The scriptPubKey is a short computer program, and the
scriptSig is the data that causes that program to return true. While the
details are not relevant here, most standard transactions use scriptPubKeys
that essentially check a signature computed over the hash of part of the
\emph{spending} transaction.


\subsubsection{Value accounting}

Talk about how to make assets divisible without fixing the satoshis/asset
ratios. Note fully fractional scheme, and powers of two scheme. (write some
quick code for powers of two?)



\subsubsection{Mapping schemes}

like "best fit" where colors are somehow fitted together, not sure I really
understand that 100\%, complex! can we take a scheme actually proposed and find
a flaw? good instructional case


\subsubsection{txout mapping with nSequence}

use nSquence on txins to send colored coins to txouts

note how this is compatible with other schemes in that the CTxIn is determinging where coins go


\subsubsection{CTxOut nValue schemes}

use low-order bits to specify colors

not as compatible with other schemes because of nValue conflicts, though they
can be detected and avoided case-by-case (e.g. coinjoin)



\subsubsection{Committed contracts}

In a txout, commit to the contract specifying where colored coins came and went.

Tricky to get right, must be sure that every txout is not hiding a contract.
Can be used with P2SH.


\subsubsection{Encrypted transfers}

use data hiding techniques to hide where colored coins were actually being sent
(re: contracts, re: nSequence?) though nSequence is still set - std txs don't
do that. Still, nice in conjunction with coinjoin to make blacklists unweildly.



\subsection{Blockchains}

talk more about merge-mining

non-pow-merge mining such as timestamped chains and proof-of-sacrifice chains

use zookeyv system as an example re: incentives to reveal data. Remember that
the incentive to reveal your blockchain data is because you want other people
to build on it - problematic in non-linear DAG schemes.

move blockchains section to beginning, and make this a true survey of all
decentralized consensus systems? talk about incentives, for instance lack of in
apps like timestamping. example: DIANNA (?)

\subsubsection{proof-of-work}

hashcash, sha256, scrypt, talk about ASIC resistance

momentum proof-of-work scheme, and why it's not asic resistant.

is there a proof that a memhard scheme can't be asymmetricly verifiable? note
gmaxwell's interactive memhard tree scheme


\subsection{verification/consensus schemes vs. fraud proofing vs. punishment after the fact}

IE consensus vs. fidelity bonded bank model vs. trusted ledgers

SCIP should get some mention too


\appendix{Censorship resistant data-hiding}

https://bitcointalk.org/index.php?topic=265488.msg3377058#msg3377058 <- turning data into valid pubkeys
https://bitcointalk.org/index.php?topic=308483.0;all <- general encryption

\bibliographystyle{plain}
\bibliography{paper}

\end{document}
